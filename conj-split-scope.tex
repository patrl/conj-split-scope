\documentclass[cronos]{ling-paper}
\usepackage{float}
\usepackage{caption}
\captionsetup{
  font=footnotesize,
  singlelinecheck=false,
  justification = centering,
  labelformat = parens,
  position = top,
  belowskip = 10pt
}
\addbibresource[location=remote]{/home/patrl/repos/bibliography/elliott_mybib.bib}

\title{Continuation semantics\\
  and the split-scope signature}
\subtitle{A note on Hirsch's (2017) argument that type-flexibility is insufficient}

\begin{acronym}
\acro{sfa}{Scopal Function Application}
\acro{fa}{Function Application}
\acro{pfa}{Pointwise Function Application}
\acro{sfa}{Scopal Function Application}
\acro{wco}{Weak Crossover}
\acro{ScoT}{Scope Transparency}
\acro{bti}{Binding Theory of Intensionality}
\acro{sti}{Scope Theory of Intensionality}
\acro{lf}{Logical Form}
\acro{qr}{Quantifier Raising}
\acro{gq}{Generalized Quantifier}
\acro{npi}{Negative Polarity Item}
\acro{doc}{Double Object Construction}
\acro{dp}{Determiner Phrase}
\acro{qp}{Quantificational Phrase}
\acro{cr}{Conjunction Reduction}
\acro{vp}{Verb Phrase}
\end{acronym}

\renewcommand*{\acsfont}[1]{\textsc{#1}}

\author{Patrick D.\,Elliott\\Massachusetts Institute of Technology}

\usepackage{appendix,float}

\begin{document}

\maketitle

\citet{hirschThesis} argues that \textit{split scope} readings of conjunction
can't be properly accounted for by positing a type-flexible meaning for
\textit{and}, and conversely that \ac{cr} \textit{must} be available in order to
account for the relevant data. In this note, we show that this argument doesn't
go through, once we look beyond a movement-oriented theory of scope-taking -- concretely, \citeauthor{barker2002}'s (\citeyear{barker2002}) analysis of conjunction
within \textit{continuation semantics} (see also \citealt[chapter
7]{barkerShan2015}) \textit{predicts} the split-scope signature, by virtue of
independently motivated type-shifting operations.

\section{Hirsch's argument}

Hirsch's argument revolves around the sentence in (\ref{hirsch1}). The key
observation is that (\ref{hirsch1}) has a reading which entails that there are
two things that John refused to do: (i) \textit{visit any city in Europe} and
(ii) \textit{visit any city in Asia}. Under this reading, \textit{and} scopes
above \textit{refuse}, but the material in each conjunct must scope below
\textit{refuse}, since each conjunct contains an \ac{npi} licensed by \textit{refuse}.

\ex
John refused to visit [any city in Europe] and [any city in
Asia].\label{hirsch1}\hfill(\citealt[p.\,90]{hirschThesis})\\
\phantom{,}\hfill $\wedge > \ml{refuse} > \ml{any}$
\xe

\citeauthor{hirschThesis} characterizes the \enquote{split-scope signature} as
follows:\footnote{As noted by \citeauthor{hirschThesis}, \citet{parteeRooth}
  already recognized the possibility split-scope readings of \textit{dis}junction.}

\ex \textsc{The split-scope signature}\\
\textit{And} scopes above some operator, which the apparent \ac{dp} conjuncts
scope below.\hfill\citep[90]{hirschThesis}
\xe

\citeauthor{hirschThesis}'s account of the split-scope signature is to derive
(\ref{hirsch1}) from the \ac{lf} in (\ref{hirsch-lf}), via deletion mechanisms.
The details are unimportant for our purposes, but it should be clear enough how
the \ac{lf} in (\ref{hirsch-lf}), if available, accounts for the split-scope signature:

\ex
John $λ x$ $\begin{aligned}[t]
  &\text{[\sub{\&P} [\sub{$v$P$_{1}$} $x$ refused [ any city in Europe
    $\lambda y$ \textsc{pro} to visit $y$]}\\
  &\text{and [\sub{$v$P$_{2}$} $x$ refused [ any city in Asia $λ z$ \textsc{pro}
    to visit $z$]]]]}
  \end{aligned}$\label{hirsch-lf}
\xe

\citeauthor{hirschThesis} shows in detail that the following denotation for
\textit{and} (a special case of \citeauthor{parteeRooth}'s
\citeyear{parteeRooth} recursive definition) can't capture this reading, on the
assumption that (\ref{hirsch1}) in fact involves \ac{qp} conjunction:

\ex
$\eval{and\sub{Q}} ≔ λQ₁Q₂P . Q₁ P ∧ Q₂ P$\hfill$\type{((e → t) → t) → ((e → t) → t) → (e → t) → t}$
\xe

Of particular interest to us, is that \citeauthor{hirschThesis} considers (and
ultimately dismisses), a derivation of the split scope reading involving
\textit{semantic reconstruction}. The \ac{lf} is sketched in figure \ref{fig:sem-recon}. Achieving split-scope via semantic reconstruction relies on some key assumptions: (i) the \acp{qp} can be type-lifted (via Montague \textsc{lift}) prior to conjunction, (ii) \ac{qr} can leave behind a higher-type ($\type{(e → t) → t}$) intermediate trace.

\begin{figure}[H]
  \centering
\caption{Split-scope via semantic reconstruction}\label{fig:sem-recon}
\begin{forest}
  [{$\begin{aligned}[t]
      &\ml{j refuse} (∃x[\ml{city-in-eu} ∧ \ml{j visit }x])\\
      &∧ \ml{j refuse} (∃y[\ml{city-in-asia} y ∧ \ml{j visit }y])
      \end{aligned}$}
  [{$λ f . \begin{aligned}[t]
      &f ((λ P . ∃x[\ml{city-in-eu} x]))\\
      &∧ f (λ P . ∃y[\ml{city-in-asia} y])
      \end{aligned}$}
    [{$λ f . f (λ P . ∃x[\ml{city-in-eu} x])$} [{any city in Europe$^{↑}$},roof]]
    [{...}
      [{and}]
      [{$λ f . f (λ P . ∃y[\ml{city-in-asia} y])$} [{any city in Asia$^↑$},roof]]
    ]
  ]
  [{$λ Q . \ml{j refused} (Q (λ x . \ml{j visit }x))$}
    [{$λ Q$}]
    [{...}
      [{John}]
      [{...}
        [{refused}]
        [{...}
          [{$Q$}]
          [{...}
            [{$λ x$}]
            [{...}
              [{\textsc{pro} visit $t_{x}$},roof]
            ]
          ]
        ]
      ]
    ]
  ]
  ]
\end{forest}
\end{figure}

Since \citeauthor{hirschThesis} is working in a framework in which
scope-taking entails \textit{movement}, \citeauthor{hirschThesis} argues that
this mechanism should be independently ruled out, since semantic reconstruction
doesn't seem to be available in the absence of syntactic reconstruction.
Evidence for this comes from the fact that, when an A-moved constituent
reconstructs for scope, it necessarily also syntactically reconstructs, as
diagnosed by condition C effects (see \citealt{romero1998}).\todo{Add the other
  references here.}

\pex
\a{}[A new theory by him$_{1}$] seems to Quine$_{1}$ [\_\_ to be needed].
\a\ljudge{*}{}[A new theory by Quine$_{1}$] seems to him$_{1}$ [\_\_ to be needed].\\
\phantom{,}\hfill(\citealt[p.\,95]{hirschThesis})
\xe

The argument against this derivation only goes through if scope-taking
necessarily involves syntactic movement. In the next section, we'll overview a prominent theory of scope taking -- \textit{continuation semantics} -- where this scope-taking doesn't necessitate movement.

\section{Enter \textit{continuations}}

\subsection{Background}

\citeauthor{hirschThesis}'s argument takes the standard-bearer for a
type-flexible theory of conjunction to be \citeauthor{parteeRooth}'s recursive
definition. Subsequently, however, a generalization of
\citeauthor{parteeRooth}'s account has been proposed by \citet{barker2002} and
\citet{barkerShan2015} within the framework of \textit{continuation semantics}.
Continuation semantics provides a perspective on scope-taking that generalizes
\citeauthor{montague1973}'s account of quantification, taking inspiration from
work in computer science on using delimited continuations to model control flow
(see, e.g., \citealt{danvyFilinski1992,wadler1994}). For our purposes,
continuation semantics amounts to the following conjecture: composition of
\textit{scopal} meanings is accomplished \textit{in-situ}, mediated by three
type-flexible operations -- \textit{lift} ($↑$), \textit{lower} ($↓$), and
\ac{sfa} ($\ml{S}$).

The definitions of lift/lower are straightforward, so we provide those first.
Lift, defined in (\ref{def:lift}), is simply a type-flexible version of \citeauthor{partee1986}'s
(\citeyear{partee1986}) Montague \textsc{lift}. Lower, defined in (\ref{def:lower}) is
simply an instruction to feed a function $m$ the identity function over
truth-values as its argument. In a continuized fragment, the role of lift is to
allow non-scope-takers to compose with scope-takers, and the role of lower is to
get an ordinary value back from a scopal value.

\pex
\a Lift (def)\\
$a^{↑}  ≔ λ k . k a$\hfill$↑:\type{a → (a → t) → t}$\label{def:lift}
\a Lower (def)\\
$m^{↓} ≔ m \ml{id}$\hfill$↓:\type{((t → t) → t) → t}$\label{def:lower}
\xe

The definition of \ac{sfa} is a little more complicated. Informally,
(\ref{cfa}) says that \ac{sfa} takes a \textit{scopal} function and a
\textit{scopal} argument (in either order), unwraps the contained values and
performs \ac{fa} on them, while sequencing the quantificational parts from left-to-right.\footnote{The careful reader will notice that $\ml{S}$ in (\ref{cfa}) has \textit{two} type-signatures. This is because the $\ml{S}$ is defined in terms of vanilla \acf{fa} -- $\ml{A}$. As in \cite{heimKratzer1998}, $\ml{A}$ is overloaded in order to account for the possibility that the function appear to the left of its argument or vice versa.
}

\ex
\acf{sfa} (def.)\\
$m \ml{S} n ≔ λ k . m (λ x . n (λ y . k (x \ml{A} y)))$\\
\phantom{,}\hfill$\ml{S}:\begin{aligned}[t]
  &\ml{S}: \type{(((a → b) → t) → t) → ((a → t) → t) → (b → t) → t}\\
  &\ml{S}: \type{((a → t) → t) → (((a → b) → t) → t) → (b → t) → t}
  \end{aligned}$\label{cfa}
\xe

A \ac{qp} such as \textit{every boy} in object position may compose \textit{in-situ} via \ac{sfa}; non-scope-takers must be systematically lifted. The scope of the \ac{qp} is evaluated by saturating the continuation argument with the identity function (lower). This is illustrated for the simple sentence \enquote{Mary likes every boy} in figure \ref{fig:cont1}. The type of the \enquote{contained} value is highlighted -- on an intuitive level, continuation-semantic composition involves surrounding meanings with additional scopal scaffolding, in order to allow them to play nicely with scope-takers.

\begin{figure}[H]
\centering
\caption{In-situ composition in continuation semantics: \enquote{Mary likes every boy}}\label{fig:cont1}
\begin{forest}
  [{$∀b[\ml{m likes} b]$}
  [{$∀b[\ml{id} (\ml{m likes} b)]$},edge label={node[midway,left,font=\scriptsize]{equiv.}}
  [{$λ k . ∀b[k (\ml{m likes} b)]$\\$\ml{S}$},edge label={node[midway,left,font=\scriptsize]{$↓$}}
    [{$λ k . k \ml{m}$} [{Mary},edge label={node[midway,left,font=\scriptsize]{$\uparrow$}}]]
    [{$λ k . ∀b[k (λ y . y \ml{likes} b)]$\\$\ml{S}$}
      [{$λ k . k \ml{likes}$} [{likes},edge label={node[midway,left,font=\scriptsize]{$\uparrow$}}]]
      [{$λ k . ∀b[k b]$} [{every boy},roof]]
    ]
  ]]]
\end{forest}
%
\begin{forest}
  [{$\type{t}$}
    [{$\type{(\hl{t} → t) → t}$\\$\ml{S}$},edge label={node[midway,left,font=\scriptsize]{$↓$}}
      [{$\type{(\hl{e} → t) → t}$} [{$\type{\hl{e}}$},edge label={node[midway,left,font=\scriptsize]{$\uparrow$}}]]
      [{$\type{(\hl{(e → t)} → t) → t}$\\$\ml{S}$}
        [{$\type{(\hl{(e → e → t)} → t) → t}$} [{$\type{\hl{e → e → t}}$},edge label={node[midway,left,font=\scriptsize]{$\uparrow$}}]]
        [{$\type{(\hl{e} → t) → t}$}]
      ]
    ]
  ]
\end{forest}
\end{figure}

\subsection{Conjunction in continuation semantics}

Within continuation semantics, conjunction receives an extremely natural
definition, given in (\ref{def:conj}); \textit{and} takes two scope-takers as
$m$ and $n$, and a \textit{continuation argument} $k$, and feeds $k$ into $m$
and $n$. As shown in detail by \citeauthor{barker2002} this subsumes all of the
cases covered by \citeauthor{parteeRooth}'s generalized conjunction.

\ex
Conjunction in continuation semantics (def.)\\
$\eval{and} ≔ λ mnk . n k ∧ m k$\hfill$\type{((a → t) → t) → ((a → t) → t) → (a → t) → t}$\label{def:conj}
\xe

In continuation semantics, the types we're dealing with will very quickly become rather unwieldy. In order to address this purely notational issue, we'll abbreviate scopal types as in (\ref{scopal-type}). This means that we can alternatively write the type signature for conjunction as $\type{C a → C a → C a}$.

\ex
$\type{C a ≔ (a → t) → t}$\label{scopal-type}
\xe

It's obvious the entry for \textit{and} in (\ref{def:conj}) will be able to account for \ac{qp} conjunction. We briefly show how it accounts for, e.g., proper name and predicate conjunction in figure \ref{fig:conj-examples}. In general, unless the coordinands are scope-takers, they must first be lifted before composing with \textit{and} via vanilla \ac{fa}.

\begin{figure}[H]
  \centering
\caption{Proper name conjunction (\enquote{John and Mary left})and predicate conjunction (\enquote{John swan and ran}).}\label{fig:conj-examples}
\begin{forest}
  [{$\ml{left j} ∧ \ml{left m}$}
  [{$λ k . k (\ml{left j}) ∧ k (\ml{left m})$\\$\ml{S}$},edge label={node[midway,left,font=\scriptsize]{$↓$}}
  [{$λ k . k \ml{j} ∧ k \ml{m}$\\$\ml{A}$}
    [{$λ k . k \ml{j}$} [{John},edge label={node[midway,left,font=\scriptsize]{$\uparrow$}}]]
    [{$\ml{A}$}
      [{and}]
      [{$λ k . k \ml{m}$} [{Mary},edge label={node[midway,left,font=\scriptsize]{$\uparrow$}}]]
    ]
  ]
    [{$λ k . k (λx . \ml{left} x)$} [{left},edge label={node[midway,left,font=\scriptsize]{$\uparrow$}}]]
  ]]
\end{forest}
%
\begin{forest}
  [{$\ml{swam j} ∧ \ml{ran j}$}
  [{$λk . k (\ml{swam j}) ∧ k (\ml{ran j})$\\$\ml{S}$},edge label={node[midway,left,font=\scriptsize]{$↓$}}
    [{$λk . k \ml{j}$} [{John},edge label={node[midway,left,font=\scriptsize]{$\uparrow$}}]]
    [{$λ k . k (λ x . \ml{swam} x) ∧ k (λx . \ml{ran} x)$\\$\ml{A}$}
      [{$λ k . k (λ x . \ml{swam} x)$} [{swam},edge label={node[midway,left,font=\scriptsize]{$\uparrow$}}]]
      [{$\ml{A}$}
        [{and}]
        [{$λ k . k (λx . \ml{ran} x)$} [{ran},edge label={node[midway,left,font=\scriptsize]{$\uparrow$}}]]
      ]
    ]
  ]]
\end{forest}
\end{figure}

\subsection{Accounting for the split-scope signature}

Now that we understand how continuation semantics handles coordination, we're in a position to see how it accounts for the split-scope signature straightforwardly, without positing movement. As with the putative \ac{lf} involving semantic reconstruction, the crucial step will be to allow \acp{qp} to be lifted (this is
independently necessary to account for inverse scope readings). We'll make a number of simplifying assumptions
here, but nothing in the analysis will crucially hinge upon them. For example,
we'll treat \ac{npi} \textit{any} simply as an existential quantifier, which is
licensed as long as it takes scope within the argument of \textit{refuse}.

The first step is to lift \textit{any European city} and \textit{any Asian
  city}, returning two higher-order scope-takers. Since \textit{and} can conjoin
anything scopal, it can conjoin the resulting meanings. The result is shown in
the figure below:

\begin{figure}[H]
  \centering
  \caption{Lift the \acp{npi} and conjoin}
\begin{forest}
  [{$λ l . l (λ k₂ . ∃y[ \ml{european-city} ∧ k₂ y])) ∧ l (λ k₁ . ∃x[\ml{asian-city} ∧ k₁ x])$\\\ac{fa}},draw=red
    [{$λ l . l (λ k₂ . ∃y[ \ml{european-city} ∧ k₂ y])$}
    [{$λ k₂ . ∃y[ \ml{european-city} ∧ k₂ y]$},edge label={node[midway,left,font=\scriptsize]{$\uparrow$}} [{any European city},roof]]]
    [{\acs{fa}}
      [{and}]
      [{$λ l . l (λ k₁ . ∃x[\ml{asian-city} ∧ k₁ x])$}
      [{$λ k₁ . ∃x[\ml{asian-city} ∧ k₁ x]$},edge label={node[midway,left,font=\scriptsize]{$\uparrow$}} [{any Asian city},roof]]]
    ]
  ]
\end{forest}
\end{figure}

The result of conjoining the lifted \acp{npi} is a higher-order scope taker. In
order to get this to compose with the rest of the embedded clause, we lift
\textit{visit} twice, and do higher-order \ac{sfa}. Informally, the result is
that the semantic contribution of \textit{visit} is distributed between the two
conjunctions. Next, we \textit{internally lower} the result, in order to set the
scope of the \acp{npi}.

\begin{figure}[H]
  \centering
  \caption{Compose the embedded clause and internally lower}
  \begin{forest}
    [{$λ l . \begin{aligned}[t]
        &l (∃y[ \ml{european-city} ∧ \ml{visit} y]))\\
        &∧ l (∃x[\ml{asian-city} ∧ \ml{visit} x])
        \end{aligned}$},draw=red
    [{...}
    [{\textsc{pro}$^{↑₂}$}]
    [{$λ l . \begin{aligned}[t]
        &l (λ k₂ . ∃y[ \ml{european-city} ∧ k₂ (\ml{visit} y)]))\\
        &∧ l (λ k₁ . ∃x[\ml{asian-city} ∧ k₁ (\ml{visit} x)])
        \end{aligned}$\\$⊛$}
      [{visit$^{\uparrow_{2}}$}]
      [{$λ l . \begin{aligned}[t]
          &l (λ k₂ . ∃y[ \ml{european-city} ∧ k₂ y]))\\
          &∧ l (λ
          k₁ . ∃x[\ml{asian-city} ∧ k₁ x])
          \end{aligned}$},fill=yellow]
    ]]]
  \end{forest}
\end{figure}

Now we compose the result with \textit{refuse} and the subject via \ac{sfa}, and lower the result.

\begin{figure}[H]
\centering
\caption{Compose the matrix clause and lower}
\begin{forest}
  [{$\ml{j refuse} ∃y[ \ml{european-city} ∧ \ml{j visit} y] ∧ ∃x[\ml{asian-city} ∧ \ml{visit} x]$},draw=red
  [{$⊛$}
    [{John$^{↑}$}]
    [{$⊛$}
      [{refuse$^{\uparrow}$}]
      [{$λ l . l (∃y[ \ml{european-city} ∧ \ml{visit} y])) ∧ l (∃x[\ml{asian-city} ∧ \ml{visit} x])$},fill=yellow]
    ]
  ]]
\end{forest}
\end{figure}

Zooming out, the graph of the semantic computation is given below:

\begin{figure}[H]
  \centering
  \begin{forest}
    [{$\type{t}$}
    [{$\type{C t}$},edge label={node[midway,left,font=\scriptsize]{$↓$}}
      [{John$^{↑}$}]
      [{$\type{C (e → t)}$}
        [{refused$^{↑}$}]
        [{$\type{C t}$}
        [{$\type{C (C t)}$},edge label={node[midway,left,font=\scriptsize]{$↓↓$}}
          [{\textsc{pro}}]
          [{$\type{C (C (e → t))}$\\$⊛$}
            [{$\type{C (C (e → e → t))}$\\visit$^{↑}$}]
            [{$\type{C (C e)}$}
              [{$\type{C (C e)}$} [{any European city},edge label={node[midway,left,font=\scriptsize]{$\uparrow$}}]]
              [{$\type{C (C e) → C (C e)}$}
                [{$\type{C a → C a → C a}$\\and}]
                [{$\type{C (C e)}$} [{any Asian city},edge label={node[midway,left,font=\scriptsize]{$\uparrow$}}]]
              ]
            ]
          ]
        ]
        ]
      ]
    ]]
  \end{forest}
\end{figure}

It's worth noting that \citet{hirschThesis} does briefly consider and ultimately
dismiss a similar solution, which involves type-lifting the \acp{npi}, and
allowing \acs{qr} of the resulting complex quantifier formed by
\textit{and}\sub{Q} to leave behind a higher type (i.e., a quantificational
trace), thus giving rise to semantic reconstruction. \citeauthor{hirschThesis}
ultimately rejects this solution, on the basis that such a derivation
disentangles syntactic position and scope, which has been argued to be
undesirable based on evidence that semantic reconstruction of a moved expression
feeds condition C. This consideration is completely irrelevant here, since
scope-taking does not logically entail \textit{movement} in continuation semantics.

\subsection{An unattested scope reading}

\ex
Some company hired a maid and a cook.\hfill\cmark $∃ > ∧$; \xmark $∧ > ∃$
\xe

\section{\textit{Only} and the split-scope signature}

\citet{hirschThesis} makes a parallel argument for split-scope for pre-DP
\textit{only}, intended to rule out the possibility that \textit{only} can
compose directly with a DP. The examples \citeauthor{hirschThesis} considers are
those like the following:

\ex
You're only required to read three books.\hfill $\ml{only} > ◻ > ∃$
\xe

In this section, I'll show that the analysis of
\textit{and} in continuation semantics (\citealt{barker2002}, \citealt[chapter 7]{barkerShan2015}), which involves re-executing a continuation, can be extended in a reasonably
straightforward fashion to \textit{only}, thus accounting for the
cross-categorial nature of \textit{only}, and \textit{predicting} split-scope by
virtue of the availability of \textit{lift}.

\subsection{Compositional focus semantics}

Here, we make concrete our assumptions concerning the consequences of
\textsc{f}-marking for the compositional semantics. We'll mostly follow \citet{Rootha}, but with a slightly
different compositional regime, inspired by \cite[chapter 5]{Charlowc}. We'll
assume that semantic values are \textit{pairs}, consisting of the (a) the
ordinary-semantic value, and (b) the focus-semantic value. This is captured by
the focus type-constructor $\type{F}$, given in (\ref{def:foc}). Some example
focus semantic values are given in (\ref{ex:foc}).\footnote{We use the
  metalanguage function $\ml{alt}$ here as a \enquote{black box}, intended to
  stand in for your favorite theory of alternatives.

  On this way of compositionalizing Roothian focus semantics, it's even possible
  to assign the focus feature \textsc{f} a denotation directly:

  \ex
  $\eval{\textsc{f}} ≔ λ x . (x,\set{x'| x' ∈ \ml{alt} x})$\hfill$\type{a → F a}$
  \xe

  Although we abstract away from the details, composition in this framework can
  proceed by lifting non-{\sc f}-marked things into pairs, where the
  focus-semantic value is just a singleton set, and composition proceeds via
  \ac{fa} of the ordinary-semantic values, and \ac{pfa} of the focus-semantic
  values. See \cite[chapter 5]{Charlowc}
}

\ex
$\type{F a ≔ (a, \set{a})}$\label{def:foc}
\xe

\pex\label{ex:foc}
\a $\eval{John} = (\ml{j},\set{j})$\hfill$\type{F e}$
\a
$\eval{John\sub{F}} = (\ml{j}, \set{x | x ∈ \ml{alt} \ml{j}})$\hfill$\type{F e}$
\a
$\eval{[\sub{VP} hug John\sub{F} ]} = (λ y . y \ml{hug j}, \set{λ y . y \ml{hug} x | x ∈ \ml{alt} \ml{j}})$\hfill$\type{F (e → t)}$
\xe

On this compositionalization, it's natural to treat \textit{only} as seeking a
\textit{scope-taker} associated with some alternatives, and returning a plain
scope-taker.\footnote{
\todo[inline]{Something about how we abstract away here from the presupposition
  vs. assertion of only sentences}} We give the denotation we assume for
\textit{only} in (\ref{def:only}):

\ex
$\ml{only} (m,\mathbb{m}) ≔ λ k . m k ∧ ∀m' ∈ \mathbb{m}[(m k ↛ m' k) → ¬ (m' k)]$\\
\phantom{,}\hfill$\type{F (C a) → C a}$\label{def:only}
\xe

n order to feed \textit{only} in the kind of meanings it's looking for, we'll
need to lift \textit{lift}, introduced in the previous section, into a function
on pairs or \textsc{o}-semantic and \textsc{o}-semantic values.

\ex
$(x,𝕪)^{↑_{F}} ≔ (x^{↑},\set{y^{↑} | y ∈ 𝕪})$
\xe

Let's now see how this entry for \textit{only} derives its flexibility:

\pex
\a Mary hugged only \textsc{John}.\label{ex:flex1}
\a Mary only hugged \textsc{John}.\label{ex:flex2}
\xe

Concentrating on (\ref{ex:flex1}), the derivation proceeds quite
straightforwardly -- first, the \textsc{f}-marked DP \textit{John} is lifted
into pair consisting of lifted \textit{John}, and lifted alternatives to \textit{John}.

\begin{figure}[H]
  \centering
\caption{Composition of \enquote{only \textsc{John}}}
\begin{forest}
  [{$λ k . k \ml{j} ∧ ∀Q ∈ \Set{x^{↑} | x ∈ \ml{alt} \ml{j}}[(k \ml{j} ↛ Q k) → ¬ (Q k)]$},fill=yellow
    [{only}]
    [{$\left(\ml{j}^{↑},\Set{x^{↑} | x ∈ \ml{alt} \ml{j}}\right)$}
    [{$(\ml{j},\set{x|x ∈ \ml{alt} \ml{j}})$\\John\sub{F}}]]
  ]
  \end{forest}
\end{figure}

Now that we have a scope-taker of type $\type{C e}$, composition of the rest of
the sentence proceeds straightforwardly -- \textit{lower} evaluates the scope of
\textit{only} at the sentential level.

\begin{figure}[H]
  \centering
  \caption{Scoping out \enquote{only \textsc{John}}}
  \begin{forest}
    [{$\ml{m hug j} ∧ ∀ p ∈ \set{\ml{m hug x}|x ∈ \ml{alt} \ml{j}}[(\ml{m hug j} ↛ \ml{m hug }x) → ¬ (\ml{m hug })x]$}
    [{$\ml{m hug j} ∧ ∀ Q ∈ \set{(\ml{m hug }x)^{↑}|x ∈ \ml{alt} j}[(\ml{m hug j} ↛ Q^{↓}) → ¬ (Q^{↓})]$}
    [{$λ k . k (\ml{m hug j}) ∧ ∀Q ∈ \Set{(\ml{m hug }x)^{↑} | x ∈ \ml{alt} \ml{j}}[(k (\ml{m hug j}) ↛ (Q k) → ¬ (Q k)]$\\$⊛$}
      [{Mary$^{↑}$}]
      [{$⊛$}
        [{hug$^{↑}$}]
        [{$λ k . k \ml{j} ∧ ∀Q ∈ \Set{x^↑ | x ∈ \ml{alt} \ml{j}}[(k \ml{j} ↛ Q k) → ¬ (Q k)]$},fill=yellow]
      ]
    ]
    ]]
  \end{forest}
\end{figure}

The derivation of (\ref{ex:flex2}) procedes in a completely parallel fashion -- \textsc{John} introduces focus alternatives; the \ac{vp} is lifted before it composes with \textit{only}.

\begin{figure}[H]
 \centering
\caption{Composition of \acs{vp} \textit{only}}
\begin{forest}
  [{}
  [{$⊛$}
    [{Mary$^{↑}$}]
    [{\ac{fa}}
      [{only}]
      [{$((λ y . y \ml{hug} j)^{↑}, \set{(λ y . y \ml{hug} x)^{↑}|x ∈ \ml{alt} \ml{j}})$} [{$(λ y . y \ml{hug} j, \set{λ y . y \ml{hug} x | x ∈ \ml{alt j}})$} [{hug John\sub{F}},roof]]]
    ]
  ]
  ]
\end{forest}
\end{figure}

\subsection{Split scope}

Let's assume that \textit{three\sub{F} papers} denotes a pair, consisting of its
ordinary semantic value, and its focus-semantic value.

\ex
$\eval{three\sub{F} papers} = \left(\begin{aligned}[c]
    &λ k . ∃X[|X| = 3 ∧ \ml{papers} X],\\
    &\set{λ k . ∃X[|X| = n ∧ \ml{papers} X] | n ∈ \mathbb{N}}
    \end{aligned}\right)$\hfill$\type{\ml{F} (C a)}$
\xe

We can analyze \textit{only} as looking for a scope-taker associated with some (scopal)
alternatives, and returning a scope-taker.

\ex
$\ml{only} (m,\mathbb{m}) ≔ λ k . m k ∧ ∀m' ∈ \mathbb{m}[(m k ↛ m' k) → ¬ (m' k)]$\\
\phantom{,}\hfill$\type{F (C a) → C a}$
\xe

I
\subsection{Deriving split scope}


\begin{figure}[H]
  \centering
  \caption{Graph of the derivation}
  \begin{forest}
    [{$\type{t}$}
    [{$\type{C t}$},edge label={node[midway,left,font=\scriptsize]{$\downarrow$}}
      [{$\type{C (C (t → t))}$\\require$^\uparrow$}]
      [{$\type{C t}$}
      [{$\type{C (C t)}$},edge label={node[midway,left,font=\scriptsize]{$\intLower$}}
        [{$\type{C (C e)}$\\you$^{↑₂}$}]
        [{$\type{C (C (e → t))}$}
          [{$\type{C (C (e → e → t))}$\\read$^{↑₂}$}]
          [{$\type{C (C e)}$}
            [{$\type{F (C a) → C a}$\\only}]
            [{$\type{F (C (C e))}$} [{$\type{F (C e)}$},edge label={node[midway,left,font=\scriptsize]{$\mathtt{fmap} \uparrow$}} [{three\sub{F} books},roof]]]
          ]
        ]
      ]
    ]]]
  \end{forest}
\end{figure}

\ex
$\begin{aligned}[t]
  &□ ∃X[|X| = 3 ∧ \ml{books} X ∧ \ml{you read} X]\\
  &∧ ∀p ∈ \ml{alts}[((□ ∃X[|X| = 3 ∧ \ml{books} X ∧ \ml{you read} X]) ↛ p) → ¬ p]\end{aligned}$
\xe

\printbibliography

\end{document}


%%% Local Variables:
%%% mode: latex
%%% TeX-engine: xetex
%%% TeX-master: t
%%% End:

% LocalWords:  Keny Niedzielski Schwarzchild Yasu Sudo
